%%%%%%%%%%%%%%%%%%%%%%%%%%%%%%%%%%%%%%%
% This is a modified ONE COLUMN version of
% the following template:
%
% Deedy - One Page Two Column Resume
% LaTeX Template
% Version 1.1 (30/4/2014)
%
% Original author:
% Debarghya Das (http://debarghyadas.com)
%
% Original repository:
% https://github.com/deedydas/Deedy-Resume
%
% IMPORTANT: THIS TEMPLATE NEEDS TO BE COMPILED WITH XeLaTeX
%
% This template uses several fonts not included with Windows/Linux by
% default. If you get compilation errors saying a font is missing, find the line
% on which the font is used and either change it to a font included with your
% operating system or comment the line out to use the default font.
%
%%%%%%%%%%%%%%%%%%%%%%%%%%%%%%%%%%%%%%
%
% TODO:
% 1. Integrate biber/bibtex for article citation under publications.
% 2. Figure out a smoother way for the document to flow onto the next page.
% 3. Add styling information for a "Projects/Hacks" section.
% 4. Add location/address information
% 5. Merge OpenFont and MacFonts as a single sty with options.
%
%%%%%%%%%%%%%%%%%%%%%%%%%%%%%%%%%%%%%%
%
% CHANGELOG:
% v1.1:
% 1. Fixed several compilation bugs with \renewcommand
% 2. Got Open-source fonts (Windows/Linux support)
% 3. Added Last Updated
% 4. Move Title styling into .sty
% 5. Commented .sty file.
%
%%%%%%%%%%%%%%%%%%%%%%%%%%%%%%%%%%%%%%%
%
% Known Issues:
% 1. Overflows onto second page if any column's contents are more than the
% vertical limit
% 2. Hacky space on the first bullet point on the second column.
%
%%%%%%%%%%%%%%%%%%%%%%%%%%%%%%%%%%%%%%
     \documentclass[]{deedy-resume-openfont}
 
     \begin{document}
     
%%%%%%%%%%%%%%%%%%%%%%%%%%%%%%%%%%%%%%
  %
  %     Profile
  %
  %%%%%%%%%%%%%%%%%%%%%%%%%%%%%%%%%%%%%%
  
\namesection{Souptik} {Datta}\infoline{\href{mailto:souptikdatta2001@gmail.com}{souptikdatta2001@gmail.com} | 8334012176 | 246, Bangur Park, Rishra}\\\vspace{4pt}\linksline{\descript{\href{https://www.hackerrank.com/souptikdatta2001}{HackerRank | }},\descript{\href{https://github.com/Souptik2001}{GitHub | }},\descript{\href{https://www.linkedin.com/in/souptik-datta-a10072183}{Linkedin | }},\descript{\href{https://souptik2001.github.io}{Blog | }},\descript{\href{https://souptik2001.github.io/about.html}{Portfolio}}}\underlineheader{}

%%%%%%%%%%%%%%%%%%%%%%%%%%%%%%%%%%%%%%
%
%     Education
%
%%%%%%%%%%%%%%%%%%%%%%%%%%%%%%%%%%%%%%
\section{Education}
\raggedright

    \runsubsection{Kalinga Institute Of Industrial Technology}\hspace*{\fill}  \location{April 2019 - April 2023}\\
    \descript{Bachelors Electronics And Telecommunications}\hspace*{\fill}\location{Odisha, India}\\
    GPA: 8.75\\
    \sectionsep
  

    \runsubsection{Aditya Birla Vani Bharati}\hspace*{\fill}  \location{2017 - 2019}\\
    \descript{Higher Secondary Science}\hspace*{\fill}\location{West Bengal, India}\\
    GPA: 8.9\\
    \sectionsep

%%%%%%%%%%%%%%%%%%%%%%%%%%%%%%%%%%%%%%
%
%     Skills
%
%%%%%%%%%%%%%%%%%%%%%%%%%%%%%%%%%%%%%%
\section{Skills}
\raggedright
\begin{tabular}{ l l }
\descript{Programming Languages} & {\location{C++, C\#, Python, Javascript, High-level shader language}} \\
\descript{Libraries/Frameworks} & {\location{Flask, Express}} \\
\descript{Tools / Platforms} & {\location{Unity, Linux}} \\
\descript{Databases} & {\location{MySQL}} \\
\end{tabular}
\sectionsep
%%%%%%%%%%%%%%%%%%%%%%%%%%%%%%%%%%%%%%
       %
       %     Projects
       %
       %%%%%%%%%%%%%%%%%%%%%%%%%%%%%%%%%%%%%%
       \section{Projects / Open-Source}
       \raggedright
       
           \runsubsection{\large{Inverse Kinematics Solver}}
           \descript{| \href{https://github.com/Souptik2001/IK-Solver}{Link}}\hfill \location{Unity, C\#}\\
           This is an inverse kinematic solver that uses the FABRIK algorithm. I have also implemented a collision detection algorithm with the basic FABRIK algorithm.\\
           \sectionsep
         
       
           \runsubsection{\large{Custom Shaders}}
           \descript{| \href{https://souptik2001.github.io/\#content:\textasciitilde{}:text=Shaders}{Link}}\hfill \location{Unity, C\#, HLSL}\\
           I recently started with shader writing for graphics rendering and I am really enjoying it. The ability to write shaders gives you so much control over how you want your game to look. Here are some projects on shaders that I am currently working on - Portal Shader (Teleportation mechanic also implemented) - https://github.com/Souptik2001/Portal
Water shader - https://github.com/Souptik2001/Water-Shader\\
           \sectionsep
         
       
           \runsubsection{\large{Game Projects}}
           \descript{| \href{https://souptik2001.itch.io/}{Link}}\hfill \location{Unity, C\#, HLSL}\\
           These are some of the games that I have created.\\
           \sectionsep
         
       
           \runsubsection{\large{Incentivized Tech Bin}}
           \descript{| \href{https://github.com/Souptik2001/Tech-Bin}{Link}}\hfill \location{Python, Html, CSS, JavaScript, MySQL, C++, Esp32}\\
           This project encourages people to throw garbage in their own dustbins and to empty the dustbin every day when the garbage collection truck comes. For the working procedure please check out the project GitHub repository.\\
           \sectionsep
         
       
           \runsubsection{\large{Website Design}}
           \descript{| \href{https://seasons-co.com/}{Link}}\hfill \location{HTML, CSS, JavaScript, Firebase}\\
           I created a static website for this company and also hosted it on firebase and linked it with a custom domain.\\
           \sectionsep

     \ 
     \end{document}